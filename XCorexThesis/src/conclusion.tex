\chapter{Conclusion}\label{ch:4}

\epigraph{One worthwhile task carried to a successful conclusion is better than
50 half-finished tasks }{B.C.Forbes}

	As stated in the beginning of the work, we wanted to ease and improve the way
analysis tools are developed and integrated in IDEs. Currently they are
mostly developed by creating a complex metamodel which is able to analyze the
model which we want to evaluate. This procedure can cause major problems when we
try to integrate multiple analysis tools in one platform because we have to
merge different metamodels (architectures) which also have different semantics.
Another problem is the procedure itself, it is repetitive, thus it can, as we
have seen in the case of CodePro, it actually did, cause design and
implementation problems.

	The main problem with CodePro is the lack of type safety due to the need of
generalization and ease of extensibility. Another problem is the manipulation of
magic constants in order to identify and describe the plugins that are created
based on the framework. 

	Our solution for this problems is to develop a meta-metamodel, a model which
will allow us to describe an analysis tool (metamodel), thus easing the process of 
integrated different tools. As we have seen the meta-metamodel is
implemented by using specific Java metadata programming, annotations, and we
enforce their semantics be defining an annotation processor. Also, the
annotation processor has the important role of generating the appropriate code
for the model based on the user specifications. Because eveyrhing is done at
compiletime, and not at runtime, the compiler can enforce type safety and IDEs
can provide nice intellisense which will ease us in the development process.
	
	
