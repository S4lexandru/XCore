\chapter{Introduction}\label{ch:1}

\section{Motivation}
	Software development is a very complex process in which every component of the
system must developed according to a strict set of guidelines. Every component
and its architecture is assets based on different metrics which are implemented
by software analysis tools. One major problem with this approach is that you
have to be a licensed statistical interpreter to fully understand the abstract
numbers that are ouputed by the tool. Another major problem is that in most
cases the use of the tools comes to late.\cite{tools:inCode}. And the biggest
problem yet is that we need very many metrics to properly evaluate our items,
but for this we also need a lot of metrics and wasted time in running them.
	For the latter problem there is an obvious solution: unite all of them
in a single platform. This is not an easy task as you might think. It implies
creating one or more meta-metamodels which can aggregate the metamodel of the
different tools, but this is error prone.
	CodePro, its comercial name is inCode\cite{site:inCode}, is an analysis tool
developed by the LOOSE Research group \cite{tools:inCode} that defines its own 
set of meta-metamodel, metamodel and model in order to be fully extensible and
independent from other platforms, but because of the architecture which can be 
seen in figure \ref{figure:codePro1UML}. But because of the wish of
extensibility and ease of use there are some major problems which can be easily
identified:
		\begin{description}
		\item[type safety] This feature is lost due to the fact that plugins are
are loaded and configured on runtime. Every entity or result must be cast to the
appropriate type before it can be processed. 
		\item[magic identifiers] Every plugin has its own string identifier which 
must be used if we want to apply a property computer or group builder  
		\end{description}
	
\section{Goals and Contributions}
	
	Our purpose for this thesis is to present a solution for the problems presented
in the previous sections.  We shall present an innovative tool
called XCorex which can solve the problems. It can do this by providing a flexible
meta-metamodel which integrates seamlessly with the metamodel of a generic
tool. The user can provider any number of tools which must be annotated with the
meta-metamodels provided by the framework. The integration of all the tools and
the generation of the appropriate model will be dealt by the XCore tool. Thus we
are moving all runtime checks that are done by CodePro at compilationtime and
taking advantage of the type safety rules which are enforced by compilator. 
This also provides an easy integration with the intellisense from every IDE.

	
	


\section {Organization}
	Chapter 2.  A thorough presentation of the fundamental concepts that are going
to be used in this work in order to avoid any ambiguities.	It will cover
elements regarded eclipse plugin development, meta-metamodeling, metamodeling,
modeling and how they can be implemented in Java. Also we present a state of the
art analysis tool CodePro and detail the problem we are solving.

	Chapter 3  The anatomy of XCore will reveal the solution we purpose for the 
problem. All mechanisms that allow the framework to work as intended (ease of
use and extensibility) will be explained in details.  Also we introduce a tool
that I have implemented in order to properly evaluate the tool.
	
	Chapter 4  In this chapter we summarize all the information described in
previous chapters and present the conclusions.
	
	In the last chapter, Chapter 5, I present the future work which will be done
in the development process of the tool.	
	